\documentclass[]{article} 

\usepackage{amsmath}
\usepackage{color}
\usepackage{enumitem}
\usepackage{fancyhdr}
\usepackage{fancybox}
\usepackage{fourier}

\usepackage{graphicx}


\usepackage{hyperref}



\definecolor{umn}{rgb}{0.4784314,0.0,0.09803922}
\hypersetup{colorlinks,breaklinks,
            linkcolor=umn,urlcolor=umn,
            anchorcolor=umn,citecolor=umn}

\title{EPsy 8220 PROJECT} 
\author{} 
\date{} 

\pagestyle{fancy}
\lhead{
\includegraphics[width=2cm]{/Users/andrewz/Pictures/UMN-Images/Goldy/goldyMoutU}}
\chead{ \textbf{EPsy 8220~~~~~~~~~~~~~~~~~~~~~~~~~~~~~~~~~~~~~~~~~~~~~~~~~~~~~~~~~~~~~~~~~~~~~~~~~~~~~~~~~~~~~~~~~~~~}\\
Project~~~~~~~~~~~~~~~~~~~~~~~~~~~~~~~~~~~~~~~~~~~~~~~~~~~~~~~~~~~~~~~~~~~~~~~~~~~~~~~~~~~~~~~~~~~~~\\
}
\rhead{}
\lfoot{}
\cfoot{}
\rfoot{\thepage}
\renewcommand{\headrulewidth}{0.4pt}
\renewcommand{\footrulewidth}{0pt}
\renewcommand{\headsep}{1.4cm}

\fancyput(3.25in,-4.5in){%
\setlength{\unitlength}{1in}\fancyoval(7,9.5)}


\begin{document}

\renewcommand{\labelenumi}{\alph{enumi}.} 

\maketitle 
\thispagestyle{fancy}

\noindent What constitutes a project for EPsy 8220 is somewhat open-ended. My initial vision was that it would take the form of an applied analysis for a potential paper to be submitted for publication. However, I think there are other ways that a project for this class could take shape. For example,

\begin{itemize}
\item Applied analysis
\item R package with documentation
\item Simulation study
\end{itemize}

\noindent Almost anything related to the material in the course would be appropriate. Run a proposal by me and I will consider it. The only thing I will not consider is a review of literature related to the course material as a stand-alone project. For those students doing an applied analysis, I can see two routes. The first is that you are \textbf{carrying out the analysis to be submitted for publication}. If this is the case, you should turn in the entire article you plan to submit for publication, including pertinent review of the literature, methodology, analysis, and discussion (not to exceed 25 pages). This is how I can give you the most relevant feedback. \\
\linebreak
The second route I see is that you are \textbf{carrying out an analysis that is specifically for this project}. Although this might lead to something beyond the course, in all likelihood, you plan to terminate the project at the semester's end. If that is your plan, then I will only need to see the an introduction that presents the research question(s) to be examined, methodology, analysis, and a brief discussion of results (not to exceed 10 pages).\\
\linebreak
If you need to find data for the project, here are some avenues you can pursue.

\begin{itemize}
\item College football data (\url{https://github.com/lebebr01/cfbFootball}): This dataset includes variables at the school and coach-level. You could use it to answer questions about whether a team will make a bowl game, or whether a coach will be fired.
\item Social survey data (\url{http://sda.berkeley.edu/archive.htm}): Data from surveys such as the General Social Survey (GSS), American National Election Studies (ANES), Current Population Survey (CPS), Census microdata, and many others are available here.
\item Education data (available from Minnesota Department of Education; \url{http://education.state.mn.us/mde/Data/})
\item UC Irvine Machine Learning Repository (\url{http://archive.ics.uci.edu/ml/}): All sorts of data from different settings and topics
\item Kaggle data (\url{https://www.kaggle.com}): Do a project and win a prize!
\end{itemize}

\noindent Need other ideas? Ask! For an applied analysis, you will need to clearly generate and lay out a series of research questions to be examined. You will also need to conduct appropriate statistical analyses that will allow you to address your research questions. Your project (written as a paper) should present your results clearly and concisely and should integrate statistical and substantive interpretation. It should also present summaries and conclusions that reflect on the broader implications and interpretation of your analyses.\\
\linebreak
Please present supporting tables and figures that are clearly labeled and referenced, as always, in APA style (however, unlike APA, please do not relegate them to the end of the paper). Do not turn in pages of calculations with little discussion; I am much more interested in your interpretation of the quantitative information. Similarly, resist the urge to demonstrate everything you have learned this semester; not everything is relevant for every data set.\\
\linebreak
This assignment has no single right research question, no single right way to address any particular question, nor any single right ``final model.'' Each analysis you conduct will likely suggest several other avenues to pursue. Some of these will be fruitful, others will not. I encourage you to try them and see what happens. But keep in mind that good research is guided by good research questions, not by a rote series of mechanical operations. 
\\ \linebreak
\noindent\textit{If you decide to not do an applied analysis, you need to talk with me about that in order to come to an agreement about what your project will entail.}
\\ \linebreak
Have fun and good luck!


%Analyze the data assigned to your group using logistic regression. Carefully justify the
%choice of your model. Be sure to:
%• Select only the important covariates.
%• Have appropriate categories for the categorical covariates.
%• Justify the scale of the continuous covariates in the model. (Consider using the
%method of fractional polynomials with only one term.)
%• Assess the goodness-of-fit of the suggested model.
%• Perform influence diagnostics.
%• Write the interpretation.
%You need to submit a typed report. The report should be neat and professional-looking.
%Include only the relevant output (preferably in a tabular form) in the body of the report.
%Move the computer code to the appendix and annotate it. Use plots wherever necessary.
%Don’t try to take shortcuts – I hate it. Try to do a good job.
%Due date: March 21, 2006
%See me


\end{document}
