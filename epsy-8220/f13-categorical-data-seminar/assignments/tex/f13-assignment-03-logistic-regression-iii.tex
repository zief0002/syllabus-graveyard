\documentclass[]{article} 


\usepackage{amsmath}
\usepackage{color}
\usepackage{paralist}
\usepackage{enumitem}
\usepackage{fancyhdr}
\usepackage{fancybox}
\usepackage{fourier}

\usepackage{graphicx}
\usepackage{setspace}

\usepackage{hyperref, url}



\definecolor{umn}{rgb}{0.003921569, 0.486274510, 0.658823529}
\hypersetup{colorlinks,breaklinks,
            linkcolor=umn,urlcolor=umn,
            anchorcolor=umn,citecolor=umn}
            



\title{LOGISTIC REGRESSION III} 
\author{} 
\date{} 

\pagestyle{fancy}
\lhead{
\includegraphics[width=2cm]{/Users/andrewz/Pictures/UMN-Images/Goldy/goldyMoutU}}
\chead{ \textbf{EPsy 8220---Methods for Categorical Response Data}\\
Lab \#2~~~~~~~~~~~~~~~~~~~~~~~~~~~~~~~~~~~~~~~~~~~~~~~~~~~~~~~~~~~~~~~~~~~~~~~~~~~~~~~~~~~~~~~~~~~~~~~\\
21pts.~~~~~~~~~~~~~~~~~~~~~~~~~~~~~~~~~~~~~~~~~~~~~~~~~~~~~~~~~~~~~~~~~~~~~~~~~~~~~~~~~~~~~~~~~~~~~
}
\rhead{}
\lfoot{}
\cfoot{}
\rfoot{\thepage}
\renewcommand{\headrulewidth}{0.4pt}
\renewcommand{\footrulewidth}{0pt}
\renewcommand{\headsep}{1.4cm}

\fancyput(3.25in,-4.5in){%
\setlength{\unitlength}{1in}\fancyoval(7,9.5)}


\begin{document}


\maketitle 
\thispagestyle{fancy}

\noindent In this assignment, you will continue to examine school district AYP data from the state of Minnesota by adding several other covariates to the model. Remember, your specific focus is \textit{examining the variability in AYP status for Minnesota schools and evaluating what predictors explain that variation.} 
\\
\linebreak
\noindent \textbf{Assignment Guidelines:} Paralleling the analysis we have carried out in class, conduct the statistical analyses necessary to complete this assignment. Please submit a word-preocessed document that responds to each of the questions below. Any R output in the documentation file should be typeset and graphics should be resized so that they don't take up more room than necessary. \textit{All tables and figures should be publication quality (i.e., follow the APA guidelines).}



\section*{Statistical Analyses}

\noindent\textbf{Re-scale the predictors.}

\begin{enumerate}[resume]
\item Mean center each of the three predcitors using the \texttt{scale()} function. Also, re-scale the newly centered teacher salary predictor by dividing by \$1,000 and re-scale the newly centered enrollment predictor by dividing by 100. Re-fit the three-predictor main-effects model from Lab \#2 using the centered and re-scaled predictors. Add this model to a summary table. When constructing this table, you might find it helpful to Connolly (2006) and Espenshade, Chung, \& Walling (2004). There is no need to write anything about the table right now; save your space and time for later questions. You will be adding several other models to this table. There is no need to re-create this table each time \ldots just add to it. \textbf{(2pts)} 
\end{enumerate}

\begin{enumerate}[resume]
\item How do the results (specifically the \textit{p}-values, decisions about predictors, and fit indices) from this model compare with those from the uncentered three-predictor main-effects model you fitted in Lab \#2? \textbf{(2pts)}
\end{enumerate}

\pagebreak

\noindent\textbf{Fit a series of interaction models that further examine the effect of average teacher salary.}

\begin{enumerate}[resume]
\item Fit the model that includes the interaction effect between average teacher salary and poverty. Add the results to your regression summary table. \textbf{(2pts)}
\item Fit the model that includes the interaction effect between average teacher salary and enrollment. Add the results to your regression summary table. \textbf{(2pts)}
\item Based on the three models fitted, which model will you adopt as a ``final'' model? Explain. \textbf{(2pts)} 
\end{enumerate}

\vspace{\baselineskip}
 \noindent\textbf{Examine the residuals from your ``final'' model.}

\begin{enumerate}[resume]
\item Examine a binned scatterplot of the residuals for the model you chose as your ``final'' model. What does it suggest about model--data fit. \textbf{(2pts)}
\item Examine plots of the influence measures. Identify any observations that might need to be removed. \textbf{(2pts)}
\item Remove the observations you identified in Question 7 and re-fit your ``final'' model. Compare and contrast the results from this model to those using the full data. \textbf{(2pts)}
\end{enumerate}

\vspace{\baselineskip}
\noindent\textbf{Present the results from your ``final'' model.}

\begin{enumerate}[resume]
\item Produce a plot of the fitted odds for your ``final'' model. Be sure to also add confidence envelopes. \textbf{(2pts)}
\item Write no more than a paragraph of text that will help readers interpret your plot. This should be written as if for publication. Feel free to also include any tables that might be beneficial to readers. \textbf{(3pts)}
\end{enumerate}



\pagebreak

%\section*{Writing}
% 
%Hearing of your analyses, the \href{http://education.state.mn.us/mde/index.html}{Minnesota Department of Education} (MDE) has hired you as a consultant (at a considerable daily rate given your newly developed statistical skills!) They've received a grant from the Ford Foundation to examine performance pay for teachers. Specifically they have asked you to comment on
%\begin{enumerate}
%\item Whether teacher salary seems related to districts making AYP, and
%\item How they could design a research study in the state of Minnesota to more thoroughly study whether performance pay is an effective conduit to student performance?
%\end{enumerate}
%Write a report to the MDE in which you specifically address the two issues. Be sure you are citing evidence from your analyses to substantiate your arguments. In writing your report, use accessible language (but include footnotes as necessary so that the external reviewers hired by MDE will be persuaded that you know what you're talking about!).  \textbf{(5pts)}



\end{document}

