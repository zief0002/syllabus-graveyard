\documentclass[]{article}\usepackage[]{graphicx}\usepackage[]{color}
%% maxwidth is the original width if it is less than linewidth
%% otherwise use linewidth (to make sure the graphics do not exceed the margin)
\makeatletter
\def\maxwidth{ %
  \ifdim\Gin@nat@width>\linewidth
    \linewidth
  \else
    \Gin@nat@width
  \fi
}
\makeatother

\definecolor{fgcolor}{rgb}{0.345, 0.345, 0.345}
\newcommand{\hlnum}[1]{\textcolor[rgb]{0.686,0.059,0.569}{#1}}%
\newcommand{\hlstr}[1]{\textcolor[rgb]{0.192,0.494,0.8}{#1}}%
\newcommand{\hlcom}[1]{\textcolor[rgb]{0.678,0.584,0.686}{\textit{#1}}}%
\newcommand{\hlopt}[1]{\textcolor[rgb]{0,0,0}{#1}}%
\newcommand{\hlstd}[1]{\textcolor[rgb]{0.345,0.345,0.345}{#1}}%
\newcommand{\hlkwa}[1]{\textcolor[rgb]{0.161,0.373,0.58}{\textbf{#1}}}%
\newcommand{\hlkwb}[1]{\textcolor[rgb]{0.69,0.353,0.396}{#1}}%
\newcommand{\hlkwc}[1]{\textcolor[rgb]{0.333,0.667,0.333}{#1}}%
\newcommand{\hlkwd}[1]{\textcolor[rgb]{0.737,0.353,0.396}{\textbf{#1}}}%

\usepackage{framed}
\makeatletter
\newenvironment{kframe}{%
 \def\at@end@of@kframe{}%
 \ifinner\ifhmode%
  \def\at@end@of@kframe{\end{minipage}}%
  \begin{minipage}{\columnwidth}%
 \fi\fi%
 \def\FrameCommand##1{\hskip\@totalleftmargin \hskip-\fboxsep
 \colorbox{shadecolor}{##1}\hskip-\fboxsep
     % There is no \\@totalrightmargin, so:
     \hskip-\linewidth \hskip-\@totalleftmargin \hskip\columnwidth}%
 \MakeFramed {\advance\hsize-\width
   \@totalleftmargin\z@ \linewidth\hsize
   \@setminipage}}%
 {\par\unskip\endMakeFramed%
 \at@end@of@kframe}
\makeatother

\definecolor{shadecolor}{rgb}{.97, .97, .97}
\definecolor{messagecolor}{rgb}{0, 0, 0}
\definecolor{warningcolor}{rgb}{1, 0, 1}
\definecolor{errorcolor}{rgb}{1, 0, 0}
\newenvironment{knitrout}{}{} % an empty environment to be redefined in TeX

\usepackage{alltt} 


\usepackage{amsmath}
\usepackage{color}
\usepackage{paralist}
\usepackage{enumitem}
\usepackage{fancyhdr}
\usepackage{fancybox}
\usepackage{fourier}

\usepackage{graphicx}
\usepackage{setspace}

\usepackage{hyperref, url}



\definecolor{umn}{rgb}{0.003921569, 0.486274510, 0.658823529}
\hypersetup{colorlinks,breaklinks,
            linkcolor=umn,urlcolor=umn,
            anchorcolor=umn,citecolor=umn}
            



\title{MULTINOMIAL REGRESSION II} 
\author{} 
\date{} 

\pagestyle{fancy}
\lhead{
\includegraphics[width=2cm]{/Users/andrewz/Pictures/UMN-Images/Goldy/goldyMoutU}}
\chead{ \textbf{EPsy 8220---Methods for Categorical Response Data}\\
Lab \#5~~~~~~~~~~~~~~~~~~~~~~~~~~~~~~~~~~~~~~~~~~~~~~~~~~~~~~~~~~~~~~~~~~~~~~~~~~~~~~~~~~~~~~~~~~~~~~~\\
20pts.~~~~~~~~~~~~~~~~~~~~~~~~~~~~~~~~~~~~~~~~~~~~~~~~~~~~~~~~~~~~~~~~~~~~~~~~~~~~~~~~~~~~~~~~~~~~~
}
\rhead{}
\lfoot{}
\cfoot{}
\rfoot{\thepage}
\renewcommand{\headrulewidth}{0.4pt}
\renewcommand{\footrulewidth}{0pt}
\renewcommand{\headsep}{1.4cm}

\fancyput(3.25in,-4.5in){%
\setlength{\unitlength}{1in}\fancyoval(7,9.5)}
\IfFileExists{upquote.sty}{\usepackage{upquote}}{}


\begin{document}


\maketitle 
\thispagestyle{fancy}

\noindent In this assignment, you will again use the \textit{Admissions.csv} data to examine the effect of financial aid on student success. You will primarily be creating tables and plots for publication in this lab.

\begin{itemize}
\item \textbf{Financial Aid:} \texttt{unmet}, \texttt{need}, \texttt{loan}, and \texttt{scholarship}
\item \textbf{Demographic and Social Integration:} \texttt{age19} and \texttt{onCampus}
\item \textbf{Geographic Origin:} \texttt{outstate} and \texttt{reciprocity}
\item \textbf{Academic Background:} \texttt{firstgen}, \texttt{act}, \texttt{ap}, and \texttt{remedial}
\item \textbf{Academic Performance:} \texttt{ratio}, \texttt{ccount}, \texttt{dcount}, and \texttt{wcount}
\end{itemize}

\noindent \textbf{Assignment Guidelines:} Conduct the statistical analyses necessary to complete this assignment. Please submit a word-processed document that responds to each of the questions below. Any R output in the documentation file should be typeset and graphics should be resized so that they don't take up more room than necessary. \textit{All tables and figures should be publication quality (i.e., follow the APA guidelines).}



\noindent\textbf{Description.}

\begin{enumerate}[resume]
\item Create a summary table that provides the mean and standard deviations for each of the 16 predictors conditioned on the three academic outcomes. This will be similar to Table 1 in Jones-White et al. (in review), except that it will be conditioned on academic outcome (i.e., six columns). Be sure that you are not using the names of the variables in the table unless they are descriptive of the variable being described (e.g., nobody but you knows what `firstgen' means). Also, be sure to provide predictor group headings similar to in Table 1. \textbf{(3pts)} 
\item For each of the five predictor groupings, comment on the differences in the academic outcome. \textbf{(5pts)}
\end{enumerate}

\vspace{\baselineskip}
\noindent\textbf{Use the `final' model fitted to the validation sample.}

\begin{enumerate}[resume]
\item Using the fitted model from your validation sample (see Lab 4) produce several plots of the fitted probabilities that allow the effects of different predictors to be displayed. How many you ultimately display is up to you, but create at least three different displays. \textbf{(6pts)}
\item For each plot, write a sentence or two of text that will help readers interpret your plot. This should be written as if for publication. \textbf{(6pts)}
\end{enumerate}





\end{document}

